
%%%%%%%%%%%%%%%%%%%%%%%%%%%%%%%%%%%%%%
% Chapter title 
\chapter{Functions}

% Chapter summary 
\begin{summary}
In this chapter, we will 
\begin{itemize}
    \item 
\end{itemize}
\end{summary}

% Chapter contents 
\newpage 

%%%%%%%%%%%%%%%%%%%%%%%%%
\section{Function Notation}

Input and Output

%%%%%%%%%%%%%%%%%%%%%%%%%
\newpage 
\section{Common Functions}

% Linear, Quadratic, Exponential 
\subsection*{Fundamental Functions} 
\[ f(x)=x  \quad\quad \text{Identity} \]
\[ f(x)=x^2 \]
\[ f(x)=x^3 \]
\[ f(x)=\frac{1}{x} \]
\[ f(x)=\sqrt{x} \]
\[ f(x)=e^{x} \]
\[ f(x)=\sin(x) \]
\[ f(x)=|x| \]
\[ f(x)=\frac{1}{1+e^{-x}} \]


%%%%%%%%%%%%%%%%%%%%%%%%%
\newpage 
\section{Transformations}

    \subsection{Scaling}
    \subsection{Translation}

\section{Domain and Range}

\section{Average Rate of Change}


%%%%%%%%%%%%%%%%%%%%%%%%%
\newpage 
\section{Application: Drop Distance per Time}

\vspace{15mm}
    \subsection*{Drop Distance, Part 1}
Design an experiment to answer the following question:
\begin{center}
	\textbf{How far will a ball drop in a given amount of time?}
\end{center}
Your experiment will involve collecting data, creating a model based on that data, and then testing the performance of that model with new measurements.  
Working with a small group, 
\begin{enumerate}
	\item Discuss the question and how you might answer it.
	\item Ask clarifying questions. 
	\item Draw a picture.
	\item Write a draft procedure. 
		Think through exactly what you will measure, and how you will make accurate and precise measurements.  
		Your procedure should be clear enough that you could give it to someone else and they could perform your experiment. 
	\item Do test-runs to make sure you know what you're doing and how to do it.
	\item Update your procedure. 
\end{enumerate}



\vspace{15mm}
    \subsection*{Drop Distance, Part 2}
\begin{enumerate}
	\item Collect as much data as you can.
	\item Organize your data into a table. 
\end{enumerate}
    
    
%%%%%%%%%%%%%%%%%%%%%%%%%
\newpage     
    \subsection*{Drop Distance, Part 3}
 \begin{enumerate}
	\item Create a scatter plot of your data. 
	\item Determine a well-fit curve to your data (ask for guidance). This is your model. 
	\item Use your model to make several predictions about times you have not measured.
	\item Validate your model by testing the accuracy of your predictions.  
\end{enumerate}
  
%%%%%%%%%%%%%%%%%%%%%%%%%
\newpage   
    \subsection{Cooling}
    
\section{Piecewise Functions}    

\section{Function Composition}    

\section{Application: Machine Learning} 

\section{Exercises} 