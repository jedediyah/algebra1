\newpage
\section*{Introduction}

How to be successful at learning:
\\ \vspace{5mm}

\noindent
\textbf{Attend} \\ 
Class time is valuable.  
Sometimes you will miss class, so always check-in to see what you’ve missed.  Keep your work organized and don’t miss out on opportunities for learning.  
Minimize distractions and your learning will be substantially more efficient and effective.
\\ \vspace{5mm}

\noindent
\textbf{Think} \\
Our goal is for you to learn!  Learning is a change in your long-term memory, and “memory is the residue of thought" {\small (\url{https://www.aft.org/sites/default/files/periodicals/willingham_0.pdf}) }.
We will pose ideas and demonstrate techniques, but the learning happens in your head.  
That learning is your responsibility and the way to learn is to process information critically through thought.
Consistently ask yourself questions like "how does this work?" and "when does this not work?".
\\ \vspace{5mm}

\noindent
\textbf{Practice} \\  
We practice to improve.  
Think about playing an instrument, drawing, or playing a sport.  
A common mistake in mathematics is to think that something “makes sense” so you stop practicing it.  
You will see things that seem easy and you will see things that seem challenging.  
In both cases, practice is essential.  
We may be showing you something with which you are already familiar because there is a deeper understanding to be found!  
If something seems challenging, we are practicing it so that we can move toward mastering it.  
\\ \vspace{5mm}

\noindent
\textbf{Academic Integrety} \\ 
Academic integrity is paramount. 
Academic dishonesty includes any participation in cheating, helping someone else cheat, or in any way misrepresenting your knowledge by submitting or manipulating your work or the work of others. 
