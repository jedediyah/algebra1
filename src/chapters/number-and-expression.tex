

%%%%%%%%%%%%%%%%%%%%%%%%%%%%%%%%%%%%%%
% Chapter title 
\chapter{Number and Expression}

% Chapter summary 
\begin{summary}
In this chapter, we will:
\begin{itemize}
    \item Write numbers using decimals, negatives, and fractions
    \item Add, subtract, multiply, and divide
    \item Expand and rewrite terms using exponents
    \item Find and approximate square and cube roots
    \item Simplify expressions in order of operations 
    \item Use variables to represent unknown quantities
    \item Evaluate variable expressions
    
\end{itemize}
\end{summary}

% Chapter contents 


%%%%%%%%%%%%%%%%%%%%%%%%%
\newpage 
\section{Numbers}

% Numbers 
%\subsection{Decimals}

\subsection{Negatives}
The negative of a number is its \emph{opposite}.  The opposite of \(58\) is \(-58\).  The opposite of \(-27.3\) is \(27.3\).  
% Plot the following numbers on the number line...

While not the same as subtraction, negative numbers can be viewed as being subtracted from \(0\), an idea that will be useful when we start to perform arithmetic with negative numbers.  

\subsubsection{Adding and Subtracting} 
Adding a positive number is equivalent to moving ``right'' or ``up'' the number line.  
Draw a number line diagram for each of the operations below. 
\[10 + 2 = 12  \]
%\[80+19 = 99   \]
\[ 108 + 15 = 123  \]
%\[  -8 + 3 = -5 \]
\[  -15 + 5 = -10   \]
%\[  -8 + 20 = 12 \]
Note that even when starting with a negative, adding a positive number still moves ``up''. 

Subtracting a positive number is equivalent to moving ``left'' or ``down'' the number line. 
Draw a number line diagram for each of the operations below. 
\[ 38 - 11 = 27 \]
\[  5-8 = -3  \]
\[  -7 - 4 = -11   \]
Adding a negative can be translated to subtraction.  For example, 
\[10 + (-2) = 10 -2 = 8 \]
\[  17 + (-8) = 9 \]
Subtracting a negative can be translated to addition.  For example, 
\[  22 - (-4) = 26 \]
\[ -7 - (-2) = -5 \]

\subsubsection{Multiplying and Dividing} 
There are four cases to consider: 
\[ (+)(+) \to (+)  \]
\[ (+)(-) \to (-)  \]
\[ (-)(+) \to (-)  \]
\[ (-)(-) \to (+)  \]
Justify each of these cases for multiplication.  The same justifications will hold for division.  

%%%%%%%%%%%%%%%%%%%%%%%%%
\newpage 

\begin{exercise}
	Draw a number line diagram for each of the following operations:  \\ \\
(a)	\[  12 + 7  \]
(b)	\[  8 + (-10) \]
(c)	\[  -3 - 4  \]
(d)	\[  -2 - (-7)    \]
\end{exercise}


%%%%%%%%%%%%%%%%%%%%%%%%%
\newpage 
\subsection{Fractions}
There are many ways to think about fractions: we can talk about fractions as part of a whole, or as a ratio, or rate, or as division.  
\begin{figure}[h!]
    \centering
    \includegraphics[width=0.7\textwidth]{img/FractionStrips.png}
    % Source: https://commons.wikimedia.org/wiki/File:FractionStrips.PNG 
    \caption{Fraction Strips. The width is a whole.}
    \label{fig:fraction-strips}
\end{figure}
\\
A lot of concepts we'll work with revolve around \emph{equivalent fractions}, fractions that are equivalent!
\begin{figure}[h!]
    \centering
    \includegraphics[width=0.5\textwidth]{img/fraction2-3.png}
    % Source: https://commons.wikimedia.org/wiki/File:Fraction2_3.svg
    \caption{Equivalent Fractions.  The fraction \(\frac{2}{3}\) is equivalent to the fraction \(\frac{4}{6}\).}
    \label{fig:equivalent-fractions}
\end{figure}
\\ 
% REDUCE 
\textbf{Reducing Fractions}
\\ \\ 
\emph{Reducing} is one of the most common tasks we'll see with fractions.  Reducing a fraction means to write a fraction in its lowest terms.  




%%%%%%%%%%%%%%%%%%%%%%%%%%%%%%%
\newpage 
\textbf{Adding and Subtracting Fractions}
\\ \\
Adding fractions involves combining numerators.   The example below could be read as ``five eigths plus two eigths equals seven eigths''.  
\begin{example}
	Find the following sum:
	\[\frac{5}{8}+\frac{2}{8}\]
\textbf{Solution:}
	\[\frac{5}{8}+\frac{2}{8} = \frac{5+2}{8} =  \frac{7}{8}\]
\end{example}
Subtraction is similar. 
\begin{example}
	Find the following difference:
	\[\frac{5}{8}-\frac{2}{8}\]
\textbf{Solution:}
	\[\frac{5}{8}-\frac{2}{8} = \frac{5-2}{8} = \frac{3}{8}\]
\end{example}
Adding and subtracting fractions requires a common denominator.  For example, we cannot cannot directly add or subtract thirds with fifths.  However, we can convert fractions to equivalent fractions with common denominators.  
\begin{example}
	Find the following sum: 
	\[\frac{7}{3} + \frac{2}{5}\]
\textbf{Solution:}
	\[\frac{7}{3} + \frac{2}{5} = \frac{35}{15} + \frac{14}{15} = \frac{49}{15} \]
\end{example}
To get a common denominator, you want to determine the smallest common multiple of your two denominators.  That is, the smallest number that is a multiple of both denominators.  In the example above, \(15\) is a multiple of both \(3\) and \(5\).  
\begin{remark}
 	When converting a fraction to an equivalent fraction, we can think of this manipulation as multiplying by \(1\).  For example, we coverted \(\frac{7}{3}\) to \(\frac{35}{15}\) by multiplying \(\frac{7}{3}\) by \(\frac{5}{5}\).  Importantly, \(\frac{5}{5} = 1\)  and multiplying by \(1\) does not change the \emph{value} of the original number. 
\end{remark}



%%%%%%%%%%%%%%%%%%%%%%%%%%%%%%%
\newpage 


\begin{exercise}
	Find the missing terms of these sums. 

	(a)	\[\frac{5}{12} + \frac{1}{12} = \frac{ }{12} \]
	(b)	\[\frac{ }{9} + \frac{2}{9} = \frac{13}{9} \]
	(c)	\[\frac{4}{3} + \frac{11}{3} = \text{---} \]
%	{\footnotesize \url{https://www.khanacademy.org/math/cc-fourth-grade-math/imp-fractions-2/imp-adding-and-subtracting-fractions-with-like-denominators/e/adding_fractions_with_common_denominators}}
\\ 
	{\footnotesize \url{https://www.ixl.com/math/grade-4/add-fractions-with-like-denominators-using-area-models}}
\end{exercise} 
\begin{exercise}
	Find the missing terms in these differences.  
	
	(a)	\[\frac{5}{12} - \frac{1}{12} = \frac{}{12} \]
	(b)	\[\frac{7}{9} - \frac{2}{9} = \text{---}  \]
	(c)	\[\frac{11}{3} - \frac{-4}{3} = \text{---} \]
\\ 
%	{\footnotesize \url{https://www.khanacademy.org/math/cc-fourth-grade-math/imp-fractions-2/imp-adding-and-subtracting-fractions-with-like-denominators/e/subtracting_fractions_with_common_denominators}}
	{\footnotesize \url{https://www.ixl.com/math/grade-4/subtract-fractions-with-unlike-denominators}}
\end{exercise} 
\begin{exercise}
	Find the missing terms in these sums and differences.  
	
	(a)	\[\frac{5}{6} + \frac{1}{2} = \frac{5}{6} + \frac{}{6} = \frac{}{6} \]
	(b)	\[\frac{7}{9} - \frac{2}{9} = \text{---}  \]
	(c)	\[\frac{11}{3} - \frac{4}{3} = \text{---} \]
\\ 	
	{\footnotesize \url{https://www.ixl.com/math/grade-6/add-and-subtract-fractions-with-unlike-denominators}}
\end{exercise} 
Additional Practice: 
\begin{itemize}
	\item {\footnotesize \url{https://www.ixl.com/math/grade-3/add-and-subtract-fractions-with-like-denominators}}
	\item {\footnotesize \url{https://www.ixl.com/math/grade-4/add-3-or-more-fractions-with-unlike-denominators}}
\end{itemize}





%%%%%%%%%%%%%%%%%%%%%%%%%
\newpage 
\section{Operations}

Adding, subtracting, multiplying, and dividing are all \emph{operations} on numbers.  In this section, we'll see some additional operations including exponentiation, square roots, and cube roots.  

%\subsection{Multiplying and Dividing}
\subsection{Exponents and Roots} 
\subsection{Order of Operations}
\subsection{Application: Pythagorean Theorem} 



%%%%%%%%%%%%%%%%%%%%%%%%%
\newpage 
\section{Expressions}

\subsection{Variables}
\subsection{Evaluating Expressions}
\subsection{Equivalent Expressions}
\subsubsection{Combining Like-Terms}

%%%%%%%%%%%%%%%%%%%%%%%%%
\newpage 
\section{Exercises} 

Online Practice from this Chapter ... 

Additional problems. 















